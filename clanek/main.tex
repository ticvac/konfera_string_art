\documentclass{article}

\author{Michal, Juli, Jáchym}
\title{String art}

\begin{document}
\maketitle

\section{Co je to string-art?}
\label{sec:string-art}
String-art je umělecká forma kdy se snažíme přiblížit nějakému obrázku pomocí
natahování nití v kruhu. Protože string-art je činnost veskrze praktická máme k
dispozici pouze konečně mnoho špendlíků ($n$), které se nacházejí po obvodu.

Naším cílem je mezi nimi natáhnout celkem $k$ nitek\footnote{V průběhu
	přeformulujeme naší úlohu tak aby namísto konstantního počtu nitek byla konstantní
	odchylka od originálu, které se snažíme dosáhnout.} tak aby se výsledek co
nejlépe podobal nějakému předem vybranému obrázku. Na vyřešení této úlohy jsme
zpracovali dva algoritmy.


\section{Algoritmy}
\label{sec:algoritmy}
Algoritmy v této sekci představíme a porovnáme jejich výhody a nevýhody.

\subsection{Všechny možnosti!!}
\label{ssec:vsechny-moznosti}
vygeneruje všechno a vybere n nejlepších
je na prd protože rozližuje jen binární stav
vyřešíme druhým alg
\subsection{Postupně zlepšujeme}
\label{ssec:postupne-zlepsujeme}
je super čuper duper drsnej
%miluju michala

\section{Výsledky}
\label{sec:vysledky}
obrázky and stuff

\end{document}
